\section{Introduction}

The introduction should describe the science drivers for  the Rubin LSST calibration requirements.

Papers to reference might include  the LSST overview paper, \cite{2008arXiv0805.2366I} 

%%%%%%%%%%%%%%%%%%%%%%%%%%%%%%%%%%%%%%%%%%%%%%%%%%%%%%
\section{LSST Calibration Strategy Overview}

Description of the high level strategy for LSST calibration. 
The strategy is based around directly measuring the system throughput as a function of wavelength, focal plane position, and time. 
Describe the a) Components of the calibration system, b) Measuring the System throughput, c) Characterizing the atmosphere.

References might include, e.g.  \cite{2017PASP..129k4502B}.

%%%%%%%%%%%%%%%%%%%%%%%%%%%%%%%%%%%%%%%%%%%%%%%%%%%%%%%%
\section{Instrument Signature Removal}

\subsection{Overview}
Mention that snap combine, if it happens, will happen as part if ISR. 
Details will be forthcoming. 
Sections should describe inputs, outputs, how they are obtained and all units.
Introduction should make clear that we are trying to reverse the order of the physical/electronic processes that affect our images, and get from a “raw” (what comes out of the electronics) to an image that represents how many photoelectrons were created in each pixel on the detector.

 
\subsection{Differential Non-linearity Correction}
Including how we get DNL look-up tables.

Output units ADU.


\subsection{Overscan Correction}
Output units ADU.

\subsection{Bias Subtraction}
Including how to get combined bias.

Output units ADU.

\subsection{CTI Correction}
Including how to get the CTI calibration file.

Output units ADU.

\subsection{Linearity Correction}
Including how to get the linearity table.

Output units ADU.

\subsection{Gain Normalization}
Including how to get PTC input.
This step converts from ADU to electrons and normalizes the gain to 1.0.

Output units electrons.

\subsection{Variance Plane Creation}
Including how to get read noise measurement from PTC.

Output units electrons.

\subsection{Crosstalk Correction}
Including how to measure the crosstalk matrix.

Output units electrons.

\subsection{Defect Mask and Interpolation}
Including how to get the defect map.

Output units electrons.

\subsection{Brighter/Fatter Correction}
Including how to get BF kernels.

Output units electrons.

\subsection{Dark Subtraction}
Including how to get combined darks.

Output units electrons.

% %%%%%%%%%%%%%%%%%%%%%%%%%%%%%%%%%%%%%%%%%%%%%%
\section{Illumination Corrections}
An introduction to the painful and flawed art of flat fielding.

\subsection{Overview}
Explain why do we do “flat fields”. 
We need to flatten the background as best we can.
We want to know how our instrument responds to focused light. 
We want to correct for small-scale changes in quantum efficiency (including dust shadows).
At the same time, we do not want to introduce extra signal that comes from the way we do flat-field screen illumination.  


\subsection{Background Flat Fields}
Describe the equations of the background flat (response to unfocused light) that makes up the image and sky background.  
Describe how to synthesize a background flat from monochromatic flats and a reference sky SED.

Output is surface brightness image.

\subsection{Reference Flux Flat Fields}
Describe the equations of the reference flux flat (response to focused light).  
Describe how to synthesize a reference flux flat from monochromatic flats and either a) the CBP or b) a dense dithered star field.  
Discuss the pros and cons of a) vs. b).
Also discuss how to remove the effects of lateral electric field distortions, which we do not want in the reference flux flat (including tree rings, picture frame, midline break, etc.

Output is a fluence image.

\subsection{Dust Flat Fields}
Describe how to use white light (1-2 LED) dome flats to find QE variations due to dust, and to create a simple dust flat that is just 1.0s with divots due to dust.

%%%%%%%%%%%%%%%%%%%%%%%%%%%%%%%%%%%%%%%%%%%%%%%%%
\section{Background Subtraction}

\subsection{Overview}

\subsection{Full Focal Plane Backgrounds}
Describe how to do full focal plane background subtraction. 
Describe the goals;  to remove impact of sky, zodiacal light, terrestrial light, etc.
This is performed after dividing by the background flat, which is then re-multiplied.

\subsection{Fringe Subtraction}
Describe how to do multicomponent fringe subtraction.
Includes how to compute fringe PCAs.
This is performed with the full focal plane background.


%%%%%%%%%%%%%%%%%%%%%%%%%%%%%%%%%%%%%%%%%%%%%%%%%%%%%
\section{From Background Subtracted Fluence Image to Pre-Calibrated Exposure}

\subsection{Overview}

\subsection{Initial PSF Model}

\subsection{Morphological Cosmic Ray Finding and Interpolation}

\subsection{Initial Astrometric Fit}

\subsection{Initial Photometric Fit}

%%%%%%%%%%%%%%%%%%%%%%%%%%%%%%%%%%%%%%%%%%
\section{Photometric Calibration}

\subsection{Determination}
Separate sub-section on how to compute instrumental throughputs from monochromatic flats.
Covers FGCM outline, as well as AuxTel atmospheric parameter input (TBD how best to combine auxtel + photometric atmospheric determination). 
The AuxTel reduction, analysis, etc, is a separate set of papers - these should be referenced here \cite{pstn-028}, \cite{pstn-027}.

Describe how to use the reference star network.

Describe the output products: spatially varying zeropoints and per-detector $S\_obs(\lambda)$.

\subsection{Application}
Coadding transmission curves per-cell. 
Describe how to apply chromatic corrections using an SED + $S\_obs,c(\lambda)$, i.e the coadded transmission curve per cell.


%%%%%%%%%%%%%%%%%%%%%%%%%%%%%%%%%%%%%%%%%
\section{Astrometric Calibration}

\subsection{Instrumental Model}
Dense dithered star-field (ugrizy) with GBDES to compute the per-detector instrumental astrometric model.

\subsection{Per-Detector WCS}
Describe the use of GBDES, with instrumental model and Gaia DR3(+) to compute the per-epoch instrumental model, per-star proper motions, and per-detector WCS.




	